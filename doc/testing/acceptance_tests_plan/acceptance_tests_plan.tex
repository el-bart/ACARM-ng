% A4==21x29.7[cm]
\documentclass[a4paper,12pt]{article}
% PL version encoded in UTF-8
%\usepackage{polski}
\usepackage[utf8]{inputenc}
%\usepackage{fancyhdr}
%\usepackage{longtable}

% setup margins
\usepackage{geometry}
%\geometry{verbose,a4paper,tmargin=2cm,bmargin=2cm,lmargin=2cm,rmargin=2cm}

% page numbering off
%\pagestyle{empty}

% make text nicely justified
%\sloppy
% do not leave small pieces on new sides
%\clubpenalty=10000
%\widowpenalty=10000

\usepackage{hyperref}

% use polish indentaion style
\usepackage{indentfirst}
% we'll use graphics
%\usepackage{graphicx}

%\pagestyle{fancy}

%\setlength{\textheight}{20cm}

\title{Acceptance Tests Plan for ACARM-ng project\\(version 0.1.0)}
\author{Bartosz 'BaSz' Szurgot (bartosz.szurgot@pwr.wroc.pl)}

\begin{document}

%
% macros
%
% version number
\newcommand{\versionNumber}[0]{0.1.0
}

% create new history entry
\newcommand{\historyEntry}[4]
{
#1 & #2 & #3 & #4 \\ \hline
}

% create new test case
\newcommand{\testCase}[7]
{
  \subsubsection{#1: #2}
  \begin{tabular}{ | p{3.5cm} | p{9cm} | }
    \hline
    \textbf{Reference} & #1 \\ \hline
    \textbf{Summary} & #2 \\ \hline
    \textbf{Description} & #3 \\ \hline
    \textbf{Preconditions} & #4 \\ \hline
    \textbf{Plan} & #5 \\ \hline
    \textbf{Post conditions} & #6 \\ \hline
    \textbf{Notes} & #7 \\ \hline
  \end{tabular}
}


%
% document's content
%
\maketitle


\tableofcontents


\section{Introduction}

This document describes testing plan (i.e. test cases to be executed) for the ACARM-ng application
(\url{http://www.acarm.wcss.wroc.pl}). Document has been organized in set of test cases that are to be performed.



\subsection{Test cases}
Each test case is described with the following fields:
\begin{itemize}
  \item reference number -- unique signature assigned to this test -- once assigned it cannot change later on.
  \item summary -- short summary of test (test name).
  \item description -- long description of test case -- what is it testing, what is it needed, etc...
  \item preconditions -- conditions that have to be met before executing test plan.
  \item plan -- steps, point-by-point to be executed in order to perform scan.
  \item post conditions -- final state that should be reached.
\end{itemize}



\subsection{Versioning notes}
When changing this document always change version number and append proper comment into change history.
Version number is organized as follows: X.Y.Z where X is release version indicating big changes in document,
Y is major version number indicating extending document's content (i.e. adding new test cases), Z is minor
version number indicating small fixes in document (typos, updating descriptions of test cases, etc...).



\section{Change history}
\begin{tabular}{ | c | c | l | }
  \hline
  \textbf{Date} & \textbf{Version} & \textbf{Changes} \\ \hline
  % entries goes here - keep them sorted: most recent on the top
  \historyEntry{2010.08.17}{0.1.0}{document creation}
\end{tabular}



\section{Test cases}
\subsection{Functional tests}
\testCase
{f-1}
{IP correlation.}
{Correlating alerts origin from and send to the same host (IP).}
{System is configured wilt all the filters.}
{
\begin{enumerate*}
\item Generate 150 alerts with the same source IP.
\item Send them to the system.
\item Generate 150 alerts with the same target IP.
\item Send them to the system.
\end{enumerate*}
}
{New meta-alert aggregating all the alerts from/to the same host is created for each unique host.}
{Post condition can be easily checked with a trigger or data base content's inspection.}


\testCase
{f-2}
{Reading alerts from prelude-manager.}
{Prelude-manager is one of the possible alert sources. Alerts gathered by it should be accessed by ACARM-ng as well.}
{System is configured with prelude's input.}
{
\begin{enumerate*}
\item Register new sensor for prelude-manager, to be used by ACARM-ng.
\item Configure prelude's input module of ACARM-ng to use newly registered profile.
\item Run ACARM-ng and wait for data from prelude-manager.
\end{enumerate*}
}
{New alerts are gathered by the system.}
{Post condition can be easily checked with a trigger or data base content's inspection.}


\subsection{Nonfunctional tests}
\testCase
{nf-1}
{Basic throughput}
{Test maximal number of alerts that can be accepted by the system in a second (assuming that it will be removed straight away).}
{System is up and running, configured without any filters and triggers. Persistency is configured to stubs.}
{
\begin{enumerate*}
\item{Start alerts generation}
\item{Measure number of alerts accepted in a given amount of time}
\end{enumerate*}
}
{Minimum reasonable number of alerts on a single-core machine is 1000.}
{}


\testCase
{nf-2}
{Proper memory management}
{Check if application does not leak memory when running.}
{System is up and running, configured with filters having minimum allowed time for buffering alerts for correlation.}
{
\begin{enumerate*}
\item{Start alerts generation}
\item{Assuming each alert requires 10kB of memory send alerts count that would use at least 10000 times more than total system memory, adding some time spaces to allow old entries to be removed from the queues.}
\end{enumerate*}
}
{System must not leak any memory}
{For testing purposes valgrind tool can be used.}


\testCase
{nf-3}
{Proper memory access}
{Test if application does not access memory not assigned to it.}
{System is configured using all available features.}
{
\begin{enumerate*}
\item{Test application using valgrind.}
\item{Test application using memory-debuging library like DUMA or Electric-Fence with:}
  \begin{enumerate*}
  \item aligning memory to the end of the page
  \item aligning memory to the begin of the page
  \item with checking for access to (recently) released memory pages
  \end{enumerate*}
\end{enumerate*}
}
{System cannot access memory it does not own.}
{You can switch dynamic libraries with LD\_PRELOAD variable when running application.}


\testCase
{nf-4}
{Typical throughput}
{Test number of alerts that can be accepted by the system in a second (assuming tey will be buffered for some time).}
{System is up and running, configured with all features turned on.}
{
\begin{enumerate*}
\item{Start alerts generation}
\item{Measure number of alerts accepted in a given amount of time}
\end{enumerate*}
}
{Minimum reasonable number of alerts on a single-core machine is 100.}
{Since alerts are buffered internally it is required to ensure proper amount of memory, given number of alerts passed to the system.}


\end{document}
