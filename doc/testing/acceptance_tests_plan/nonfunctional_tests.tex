\testCase
{nf-1}
{Basic throughput}
{Test maximal number of alerts that can be accepted by the system in a second (assuming that it will be removed straight away).}
{System is up and running, configured without any filters and triggers. Persistency is configured to stubs.}
{
\begin{enumerate*}
\item Start alerts generation.
\item Measure number of alerts accepted in a given amount of time.
\end{enumerate*}
}
{Minimum reasonable number of alerts on a single-core machine is 1000 per second.}
{}


\testCase
{nf-2}
{Proper memory management}
{Check if application does not leak memory when running.}
{System is up and running, configured with filters having minimum allowed time for buffering alerts for correlation.}
{
\begin{enumerate*}
\item Start alerts generation.
\item Assuming each alert requires 10kB of memory send alerts count that would use at least 10000 times more than total system memory, adding some time spaces to allow old entries to be removed from the queues.
\end{enumerate*}
}
{System must not leak any memory}
{For testing purposes valgrind tool can be used.}


\testCase
{nf-3}
{Proper memory access}
{Test if application does not access memory not assigned to it.}
{System is configured using all available features.}
{
\begin{enumerate*}
\item Test application using valgrind.
\item Test application using memory-debuging library like DUMA or Electric-Fence with:
  \begin{enumerate*}
  \item aligning memory to the end of the page
  \item aligning memory to the begin of the page
  \item with checking for access to (recently) released memory pages
  \end{enumerate*}
\end{enumerate*}
}
{System cannot access memory it does not own.}
{You can switch dynamic libraries with LD\_PRELOAD variable when running application.}


\testCase
{nf-4}
{Typical throughput}
{Test number of alerts that can be accepted by the system in a second (assuming tey will be buffered for some time).}
{System is up and running, configured with all features turned on.}
{
\begin{enumerate*}
\item Start alerts generation.
\item Measure number of alerts accepted in a given amount of time.
\end{enumerate*}
}
{Minimum reasonable number of alerts on a single-core machine is 100.}
{Since alerts are buffered internally it is required to ensure proper amount of memory for a given number of alerts passed to the system.}


\testCase
{nf-5}
{Documentation usefulness.}
{ACARM-ng installation and configuration is described in documentation. It should be checked if it's up to date and readable.}
{System is in compressed form, no component is installed nor configured yet.}
{
\begin{enumerate*}
\item Extract ACARM-ng from archive.
\item Open installation documentation (doc/INSTALL.txt) and follow instruction step by step.
\item Verify that system is installed and able to operate.
\item Open configuration documentation (doc/CONFIGURATION.txt) and follow instruction step by step, configuring all the features available.
\item Verify that all features present in the release are described in configuration.
\item Verify that installed and configured system is running correctly.
\item Check if minimal requirements (doc/REQUIREMENTS.txt) list is complete and non-redundant.
\end{enumerate*}
}
{System is installed, configured and running.}
{}
