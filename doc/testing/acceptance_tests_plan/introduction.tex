This document describes testing plan (i.e. test cases to be executed) for the ACARM-ng application
(\url{http://www.acarm.wcss.wroc.pl}). Document has been organized in set of test cases that are to be performed.



\subsection{Test cases}
Each test case is described with the following fields:
\begin{itemize*}
  \item reference number -- unique signature assigned to this test -- once assigned it cannot change later on.
  \item summary -- short summary of test (test name).
  \item description -- long description of test case -- what is it testing, what is it needed, etc...
  \item preconditions -- conditions that have to be met before executing test plan.
  \item plan -- steps, point-by-point to be executed in order to perform scan.
  \item post conditions -- final state that should be reached.
  \item notes -- additional pieces of information and hints for the tester.
\end{itemize*}



\subsection{Running tests}
For pieces of information on how to install, configure and run ACARM-ng application refer to system's documentation.

In general, all tests presented in this document should be run in ACARM-ng buil with following profiles:
\begin{itemize*}
\item release (make PROFILE=release)
\item debug (make PROFILE=debug)
\item debug with memory checking (make PROFILE=debug MEM\_DEBUG=1)
\end{itemize*}
Notice however that some of them are not relevant in some of the modes. For example it does not make sense
to measure throughput in other build than 'release'.



\subsection{Versioning notes}
When changing this document always change version number and append proper comment into change history.
Version number is organized as follows: X.Y.Z where X is release version indicating big changes in document,
Y is major version number indicating extending document's content (i.e. adding new test cases), Z is minor
version number indicating small fixes in document (typos, updating descriptions of test cases, etc...).
